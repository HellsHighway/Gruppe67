\chapter{Gesundheit}
\section{A \& Multitasking}

Technische Ger�te wie Smartphones und Laptops sind ein fester Bestandteil der Gesellschaft geworden. Laut einer Studie \cite{Cyberkrank} im Jahr 2011 besa�en 96 \% von College-Studenten im Alter von 18 bis 22 Jahren ein Smartphone und 89\% Prozent f�r die bereits Arbeitenden im gleichen Alter.  In Deutschland kann man in den der gleichen Altersgruppe �hnliches vermerken. 2011 besitzen in Deutschland 25\% ein Smartphone, 2013 sind es bereits 72\% Prozent \cite{Cyberkrank}. Dies ist nun 5 Jahre her also kann man davon ausgehen, dass die �berwiegende Mehrheit heutzutage ein Smartphone besitzt.
Digitale Medien, welche rund um die Uhr �ber das Smartphone in der Tasche leicht zu erreichen sind, regen zum Multitasking an. Dabei ist es schwierig  f�r Menschen Multitasking zu betreiben. Prof Dr. Torsten Schubert von der Humboldt\-Universit�t  behauptet dass Multitasking durch Entscheidungsprozesse zu einer erh�hten Fehlerquote oder einer verl�ngerten Bearbeitungszeit\cite{Gehirn}. 
Gem�� Schubert gilt generell: Wenn Aufgaben das gleiche Hirnareal beanspruchen, st�ren sie sich und  dadurch wird Multitasking ineffizient. Die Natur des Menschen diktiert ihm sich vollst�ndig auf eine Aufgabe zu konzentrieren um diese zum Besten seiner F�higkeiten zu erledigen. Multitasking wird mit dem Antrainieren von Sucht und Aufmerksamkeitsst�rungen gleichgestellt\cite{Gehirn}.
Jugendliche nutzen ihr Smartphone 150 Mal am Tag, folglich wird eine T�tigkeit im Durschnitt alle 6 Minuten unterbrochen \cite{iDisorder}. Die Psychologin Lydia Burak von der Bridgewater State University f�hrte  eine Umfrage, \cite{Cyberkrank} (Massachusetts, USA) die bei einer Testgruppe von 774 Studenten im Alter von 20 bis 75 Jahren und einer Verteilung der Geschlechter von 67,1\% weiblich und 32,9\% m�nnlich mittels Fragebogen feststellt, dass sich nur 5,6\% nicht noch mit anderen Aktivit�ten w�hrend der Lehrveranstaltungen besch�ftigen. Dies ist interessant, da mithilfe eines Multiple-Choice Test sofort nach einer Vorlesung, die den darin behandelten Stoff abfragt festgestellt werden konnte, dass die abgelenkten Studenten (Multitasking) schlechtere Resultate erzielten \cite{}.

Cap: H�ufigkeit korrekter Antworten im Test nach der Vorlesung in Abh�ngigkeit davon, ob die Studenten zugleich mit anderen Aufgaben am Computer selbst besch�ftigt waren oder nicht (linke Abbildung) bzw. davon, ob die Studenten anderen Studenten beim Multitasken am Laptop zuschauen konnten oder nicht (rechte Abbildung).

\section{Beeintr�chtigung der Psyche des Menschen}
Das Verwenden von Smartphones und Laptops st�rt das Lernen direkt durch Konzentrationsverlust w�hrend Lehrveranstaltungen und beeinflussen indirekt die akademische Leistung.

Eine Studie in Schweden von Biomedcentral Public Health \cite{Mobile} konnte  im Jahr 2011 mit 4156 Probanden aus der schwedischen Bev�lkerung, davon 1455 m�nnlich und 2701 weiblich, eine Relation erstens zwischen hohem Mobiltelefongebrauch und Stressempfinden, und zweitens zwischen Schlafst�rungen und Symptomen von Depression feststellen. Die Probanden sind in einem Alter von 20 -24 Jahren von welchen 40\% der M�nner und 48\% der Frauen studieren.  Eine weiter Studie in Schweden von Biomedcentral Psychiatrics \cite{Computer} im Jahr 2012 mit 4163 Probanden im gleichen Alter, davon 1458 m�nnlich und 2705 weiblich, untersucht die gleichen Folgen bei hohem Computergebrauch .
Beide Studien verbinden die verbrachte Menge an Zeit mit Mobiletelefone also die Konstante Erreichbarkeit oder vor dem Computer mit entweder einem Risiko f�r erh�htes Stressempfinden, Schlafst�rungen oder Entwicklung von Symptomen der Depression.
Desto mehr Zeit vor einem Bildschirm verbracht wird desto gr��er ist auch der Einfluss auf die Art wie der Mensch mit seinem Sozialen Umfeld interagiert. In Neuseeland im Rahmen einer Studie \cite{Adolescent} von PhD. Rosalina Richards, PhD Rob McGee, D.Sc Sheila M. Williams, PhD. David Welch and MD Robert J. Hancox wurden 1987\-1988, 976 Jugendliche im Alter von 15 Jahren befragt wieviel Zeit sie vor dem Fernseher verbringen und es wurde ihre Beziehung zu Gleichaltrigen und ihren Eltern bewertet mittels einer gek�rzten Version von \"Inventory of Parent and Peer Attachment\"(IPPA).
In der gleichen Studie [5] wurden 2004 3983 Sch�ller aus 144 verschiedenen Schulen die gleichen Fragen gestellt und als Erg�nzung wurden die Probanden gefragt wieviel Zeit sie vor dem Computer verbringen, wieviel Zeit sie mit Lesen und dem erledigen von Hausaufgaben in ihrer Freizeit verbringen.
In der ersten Phase 1987-1988 konnte man feststellen, dass jeder weiter Stunde vor dem Fernseher das Risiko eine schlechte Beziehung mit den Eltern zu haben um 13\% erh�ht und mit Gleichaltrigen sogar um 24\%. In der zweiten Phase 2004 war das Resultat �hnlich, mehr Zeit vor dem Bildschirm egal ob Fernseher oder Computer f�hrte zu einem h�heren Risiko eine schlechte Beziehung zu den Eltern oder Gleichaltrigen zu f�hren. Zeit mit Lesen und der Bew�ltigung von Hausaufgaben wurde hingegen mit einer besseren Beziehung zu den Eltern in Zusammenhang gebracht [5].
