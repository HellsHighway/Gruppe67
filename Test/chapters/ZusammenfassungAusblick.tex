\chapter{Zusammenfassung und Ausblick}

Die Politik in Kooperation mit dem  Lehrverband ist trotz allem im Stande diese H�rden zu �berw�ltigen, indem sie sich mehr mit dem Thema Digitalisierung  auseinandersetzen.  Der aktuelle Kurs  dieser weltweiten Initiative  besteht darin, die Digitalisierung in der Schule zu integrieren. Politiker glauben, dass dies der einzige Weg und zugleich der einfachste ist  zur verbesserten Bildung. Wie wir hier gesehen haben gibt es viele H�rden sei es auf einer gesundheitlicher Ebene, in der Grammatik und die Implementierung von effektiven Bildungsmethoden wie die Spielifizierung des Lernens. 
Durch Textverarbeitung haben sich Fehler in der Grammatik von 1980 auf 2006  bedeutend gesteigert. Die Industrie hat in dieser Instanz noch kein Optimum entsprechender Programme erreicht.
Laut einer Studie des deutschen Lehrerverband in Kooperation mit dem Schreibmotorik Institut  verlernen 80% der Sch�ler die Handschrift was mit regelm��igen Verwenden von Textverarbeitungssoftware korreliert. Zus�tzlich wird das Ged�chtnis �ber das  Geschriebene negativ beeinflusst.  Konventioneller Unterricht l�sst sich auf viele Art und Weisen anders gestalten. 

Diverse Techniken erm�glichen eine Integration spielerischen Elementen in die Bildung. Elemente, welche Sch�ler motivieren l�nger und effizienter zu lernen. Die Gamifikation-Design Prinzipien aus der Spielwelt motivieren beim Lernprozess unter anderem durch Rapides Feedback, Status, Flexibilit�t, sofern es gelingt diese zu implementieren, und die passende Technologie in Klassens�len vorhanden ist. Genau diese invasive Technologie ist jedoch in  bildungsgewidmeten Infrastrukturen umstritten und bringt  Nachteile mit sich.  Aus diesem Grund lenken die Rechner die Sch�ler w�hrend einer Vorlesung ab. Diese Rechner f�hren wiederum zu multitasking, was in schlechteren Testergebnissen resultiert. 
Studenten sind dauerhaft �ber digitale Ger�te wie Smartphones und Laptops  vernetzt. Dies hat einen Einfluss auf die Gesundheit und das Wohlbefinden der Gesellschaft. Vorerw�hnte Studien beweisen, dass es immer mehr abgelenkte Studenten gibt. So haben von 774 Studenten 5,6% angegeben sich nicht anderweitig w�hrend den Lehrveranstaltungen zu besch�ftigen. Au�erdem schauen Jugendliche t�glich bis zu 150 Mal auf ihr Smartphone, d. h. im Durchschnitt wird ihre Aufmerksamkeit alle 6 Minuten unterbrochen. die entweder unter einem erh�hten Stress empfinden leiden, sich von ihren Eltern und Gleichaltrigen isoliert f�hlen oder Symptome der Depression aufzeigen. All dies f�hrt zu einer Verschlechterung der akademischen Leistungen an Schulen und Universit�ten.
Als Schlusswort schlagen wir vor, eine Weiterbildung der Lehrer zu f�rdern, die es ihnen erm�glicht, auf die H�rden der Digitalisierung aufmerksam zu machen. Die Digitalisierung in Schulklassen sollte vigilant fortschreiten um die negativen Folgen vorzubeugen. 
