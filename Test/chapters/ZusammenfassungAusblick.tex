\chapter{Zusammenfassung und Ausblick}


%Der aktuelle Kurs  dieser weltweiten Initiativen  besteht darin, die Digitalisierung in die Schule zu integrieren. Politiker glauben, dass dies der einzige und zugleich der einfachste Weg ist  zur verbesserten Bildung.  
%Wie hier gesehen haben gibt es viele H�rden sei es auf einer gesundheitlichen Ebene, in der Grammatik oder die Implementierung von effektiven Bildungsmethoden wie die Spielifizierung des Lernens.


Durch Textverarbeitung haben sich Fehler in der Grammatik von 1980 auf 2006  bedeutend gesteigert. Die Industrie hat in dieser Instanz noch kein Optimum entsprechenden Textverarbeitungsprogramme erreicht. Laut einer Studie des Deutschen Lehrerverbands in Kooperation mit dem Schreibmotorik Institut  verlernen 80\% der Sch�ler die Handschrift, was mit regelm��igen Verwenden von Textverarbeitungssoftware korreliert. 

Konventioneller Unterricht l�sst sich auf viele Arten und Weisen anders gestalten, wie zum Beispiel auf spielerischer Ebene. Diverse Techniken erm�glichen eine Integration spielerischer Elemente in die Bildung, die Sch�ler motivieren, l�nger und effizienter zu lernen. 
Die Designprinzipien aus der Spielwelt motivieren beim Lernprozess unter anderem durch Rapides Feedback, Status und Flexibilit�t, sofern es gelingt diese zu implementieren und die passende Technologie in Klassens�len vorhanden ist. Genau diese invasive Technologie ist jedoch in  bildungsgewidmeten Infrastrukturen umstritten und bringt  Nachteile mit sich. Aus diesem Grund lenken die Rechner die Sch�ler w�hrend einer Vorlesung ab. Diese Rechner f�hren wiederum zu multitasking, was in schlechteren Testergebnissen resultiert. 

Studenten sind dauerhaft �ber digitale Ger�te wie Smartphones und Laptops  vernetzt. Dies hat einen Einfluss auf die Gesundheit und das Wohlbefinden der Gesellschaft. Vorerw�hnte Studien beweisen, dass es immer mehr abgelenkte Studenten gibt. So haben von 774 Studenten 5,6\% angegeben sich nicht anderweitig w�hrend den Lehrveranstaltungen zu besch�ftigen. Au�erdem schauen Jugendliche t�glich bis zu 150 Mal auf ihr Smartphone, das hei�t im Durchschnitt wird ihre Aufmerksamkeit alle 6 Minuten unterbrochen. Folgen davon sind: Verschlechterung der akademischen Leistung, ein erh�htes Stressempfinden und eine Isolation von der Gesellschaft.

Offenbar gibt es viele H�rden, sei es auf einer gesundheitlichen Ebene, in der Sprachlehre oder die Implementierung von effektiven Bildungsmethoden wie die Spielifizierung des Lernens. Diese H�rden k�nnen mit den richtigen Initiativen, in der Politik und zusammen mit den Lehrverb�nden �berw�ltigt werden.


Als Schlusswort schlagen wir vor, eine Weiterbildung der Lehrer zu f�rdern, die es ihnen erm�glicht, auf die H�rden der Digitalisierung aufmerksam zu machen. Die Digitalisierung in Schulklassen sollte vigilant fortschreiten um den negativen Folgen vorzubeugen. 

