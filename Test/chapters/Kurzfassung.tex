\kurzfassung

%% deutsch
\paragraph*{}
Traditionelle Bildung versagt. 
Die neuen Technologien k�ndigen eine neue �ra der Bildung an. Unsere Arbeit verweist auf die H�rden der Implementierung digitaler Hilfsmittel in Lehrveranstaltungen. Anfangs behandeln wir die Implementierung von Belohnungssystemen im Lernprozess, 
mit dem Resultat, dass Sch�ler motivierter sind, jedoch keine positive Korrelation zwischen Technologie und Leistung existiert. Zudem vergleichen wir die Leistung und Schreibmotorik der Old-School Sch�ler zu der von Technologie gesegneten Generation Z.
Trotz elektronischer Hilfsmittel verschlechtert sich die Grammatik. Abschlie�end zeigen wir, dass diese Technologien die Ursache f�r erh�hte Depressionserscheinungen sowie Konzentrationsmangel 
bei Jugendlichen sind und deswegen f�r eine schw�chere akademische Leistung sorgen.

