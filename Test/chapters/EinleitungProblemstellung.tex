\chapter{Einleitung und Problemstellung}

Motivations- und Anteilnahmem�ngel sind vorrangige Probleme im aktuellen Schulwesen und k�nnten durch Technologie gel�st werden. Einerseits gibt es vielversprechende Initiativen, beispielsweise bildungsorientierte Spiele zu entwerfen, um den Spa� am Lernen zu wecken. 
Andererseits erweist sich die effektive Integration spielerischer Komponenten als kostspielig und aufwendig. 
Au�erdem ist der positive Impakt durch den Einsatz neuer Technologien im Schulwesen umstritten. 
Der Bereich der Textverarbeitungsprogramme ist beispielsweise problematisch, denn meistens haben diese Programme eine W�rterkorrektur implementiert, auf die sich manche Menschen zu sehr verlassen und eventuell die F�higkeit verlieren, sich selbst �ber die Korrektheit der Eingabe zu vergewissern. 
Durch regelm��iges Verwenden dieser Software verlernen Nutzer ebenfalls die Handschrift.
\grqq{Der Mensch ist, was er als Mensch sein soll, erst durch Bildung.}" \signed{-Wilhelm Friedrich}
Die Revolution des Bildungswesen ist ein stets relevantes Thema in der Politik und Gesellschaft. 
Die gegenw�rtige Initiative in der Politik besteht darin, der Jugend das Lernen durch Einsatz neuer Software und Hardware zu erleichtern. 
E-Learning ist gem�� Studien eine verhei�ungsvolle Antwort auf Streitpunkte kontempor�rer Bildungsmechanismen. Nichtsdestotrotz bringt die Digitalisierung sowohl Risiken als auch Erfolge mit sich. Ein weiteres wichtiges Thema sind die psychologischen Abdr�cke der Technologien. Eine Digitalisierung der Bildung f�hrt automatisch zu einem vermehrten Nutzen von Computern, Smartphones und anderen technischen Hilfsmittel wie Tablets. Der Zustand der Psyche eines Studenten wirkt sich direkt auf seine akademische Leistung aus. Es ist notwendig sich mit den Auswirkungen jener Hilfsmittel mit den Studenten auseinanderzusetzen. 